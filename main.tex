
% \documentclass[12pt, twoside]{report}
\documentclass[12pt]{report}
\usepackage{mydocsetup}

%=============== START OF THE DOCUMENT
\begin{document}
\onehalfspacing

%=============== SETUP: PARAGRAPH STYLES
\setlength{\parindent}{0em} % start indentation
\setlength{\parskip}{1em} % vertical space between two paragraphs

%========= PAGE: TITLE PAGE (COVER)
\thispagestyle{empty}
\begin{titlepage}
	\begin{center}
		\vspace*{2cm}
		% \includegraphics[width=0.15\textwidth]{uoc/uocLogo.jpg}\\
		\vspace{1cm}
		{\LARGE Analysis of Thalawkelle Tea Estate PLC}
		\vspace{2cm}
		\begin{large}

			\textbf{ITC 2242 – Economics and Finance}\\
			\vspace{2cm}
			Dissanayake D.A.N.P. \\
			TE92515\\
			ICT/18/813 \\


			\vspace{2cm}
			University of Sri Jayewardenepura.\\
			Faculty Of Technology \\
			Department of ICT\\
			\vspace{2.5cm}
			\vfill
			\centering
			{\small This assignment is submitted to the University of Sri Jayewardenepura as a partial requirement of the ITC 2242 – ECONOMICS AND FINANCE}
		\end{large}
	\end{center}
\end{titlepage}

%========== PAGE: TABLE OF CONTENT
\renewcommand{\contentsname}{\LARGE{Contents}}
\setcounter{page}{2}
\pagenumbering{arabic}
% \addcontentsline{toc}{chapter}{\bf{Contents}}
\begin{singlespacing}
	\tableofcontents
\end{singlespacing}
\setlength{\parskip}{1em}
\renewcommand{\baselinestretch}{2.0}

%================ SETUP: SECTION TITLE FONT SIZES
\titleformat*{\chapter}{\fontsize{19}{22}\bfseries}
\titleformat*{\section}{\fontsize{16}{19}\bfseries\normalfont}
\titleformat*{\subsection}{\fontsize{14}{17}\bfseries\normalfont}
\titleformat*{\subsubsection}{\fontsize{12}{15}\bfseries\normalfont}

%>>>>>>>>>>>>>>>>>>>>>>>>>>>>>>>>>>>>>>>>>>>>>>>>>>>>>>>>
% Main body of Thesis
%========= Start of Thesis 
\newpage
% \setcounter{page}{1}
\setcounter{secnumdepth}{3} % enable subsubsection numbering
\onehalfspacing

\chapter{Introduction}
Talawakelle Tea Estates PLC is one of the leading plantation companies in Sri Lanka, producing high-quality pure Ceylon Tea from seventeen finest tea gardens in the island.  The company consists of its own hydroelectric plants to produce clean power, free healthcare for employees and most effective and hygienic technology in the tea production industry. Furthermore, this company pioneered the Food Factory concept, and have constantly been judged as the best among regional plantation companies in Sri Lanka.

The most recent chapter of the history of Talawekelle Tea Estate began in 1992, when Talawakelle Tea State PLC was established, following a privatization program initiated by the government of Sri Lanka.

This company a public quoted company which is major tea manufacturer in Sri Lanka who is also a leading world renown tea exporter and contribute to the Sri Lankan manufacturing sector in mass scale.  This company exports its high-quality products to the United Arab Emirates, Syria, Iran, Turkey, Jordan, Kuwait, Iraq, Japan, Libya, Chile, and Saudi Arabia, as well as CIS countries.

In order to support the mass production, this company manages multiple states located in various areas of the island. Following is a brief overview of the estates owned by the company and their growth capabilities.

\begin{table}[H]
	\centering
	\begin{tabular}{ |p{5cm}|p{5cm}| }
		\hline
		\bf{High Growth Estates} & \bf{Low Growth Estates} \\
		\hline
		Calsay                   & Moragalla               \\
		Clarendon                & Kiruwanaganga           \\
		Dessford                 & Deniyaya                \\
		Somerset                 & Handford                \\
		Radella                  &                         \\
		Bearwell                 &                         \\
		Great Western            &                         \\
		Mattakelle               &                         \\
		Holyrood                 &                         \\
		Wattegoda                &                         \\
		Palmerston               &                         \\
		Logie                    &                         \\
		\hline
	\end{tabular}
	\caption{Estates owned by Talawakelle Tea State PLC}
\end{table}


\chapter{Products}
Talawakelle Tea Estates PLC manufacture number of high-quality products in order to cater both local and non-local customers targeting both domestic and foreign markets. These products have proven to be successful and highly demanded in both markets through sales and overall performance over the years due to continues improvements and excellent customer service.

\noindent
Following is the list of products currently being manufactured by the company.

\begin{itemize}
	\item {\bf{Abbotsford Special}} \\
	      This has a special rich flavor and antioxidants and produced exclusively from buds harvested from tea bushes planted by British planter, M. Ferguson in 1871 on Abbotsford estate, through a unique manufacturing process and is packed in a metal canister.

	\item {\bf{Black Tea – BOPF Pack}} \\
	      This tea consists of a unique blend of selected BOPF from high grown estates packed in alu-foil. In 250 g and 500 g packs.

	\item {\bf{BOP- Single Origin Tea (Lose Tea)}} \\
	      This tea has six different origins - Bearwell, Dessford, Great Western, Holyrood, Mattakelle and Somerset.

	\item {\bf{BOPF – Single Origin Tea (Loose Tea)}} \\
	      Same as BOP, this tea also has six different origins - Bearwell, Dessford, Great Western, Holyrood, Mattakelle and Somerset.

	\item {\bf{Tea Bags – Single Origin Tea}} \\
	      Like BOPF, these tea bags have the same six different origins - Bearwell, Dessford, Great Western, Holyrood, Mattakelle and Somerset.

	\item {\bf{Tea Book (Gift Pack) }} \\
	      This is a selection between two tea bag packs and two loose tea packs from same six origins mentioned before – gift book.

	\item {\bf{Six in One (Gift Pack) }} \\
	      Combination of all six origins, shrink warp as one presentable pack. BOP Loose Tea, BOPF Loose Tea and Tea Bags – BOPF

	\item {\bf{Green Tea - Radella Estate}} \\
	      Finest green tea from Radella with truly splendid aura and packed in alu foil and metal canister as loose tea.
\end{itemize}

\chapter{Determinants of demand and supply}

\section{Determinants for demand}
In an industry the ability, desire and willingness to purchase a good or service is known as the demand and the factors demand depend on are known as the determinants for demand

Talawakelle Tea Estates PLC also has factors that determine the demand for their products in domestic and non-domestic markets.

\noindent
Following is the list of those determinants.
\begin{itemize}
	\item {The tastes and preferences of consumers}
	\item {The price of goods}
	\item {Increasing popularity}
	\item {Price of the competitors}
	\item {Availability of substitute good}
\end{itemize}

\subsection{The tastes and preferences of consumers}
Tea has been one of the most popular aromatic beverage since centuries. As a result of this, there is a vast number of tea variants and types. Persons tea preference can vary depending on the age, taste and other factors such as heath. Due to this high popularity of tea, customer preferences play a major role in the demand. In order to cater to this  high number of consumers companies such as Talawakelle Tea Estates PLC monitor and analyze new trends and changes in customer preferences.

\subsection{The price of goods}
The price of tea plays a major role in the demand. Since there are many tea variants with different prices it's very popular among different types of consumers regardless of their financial situations. The ability to cater to vast customer base is one of the biggest reasons why tea is very popular throughout the world.

\subsection{Increasing popularity}
The global population has been increasing rapidly. As a result of that and reasons such as sudden increase of using tea as a health product, tea has been getting popular in many types of communities. Thus, the demand for tea has been increasing significantly.

\subsection{Price of the competitors}
There is a large number of tea manufacturers and resellers in the market. As a result of that there is always a competition between them. In order to gain more consumers tea manufacturers attempt to reduce the price and improve the quality of their products compared to their competitors. This is also a factor that contributes to the demand.

\subsection{Availability of substitute good}
There are various substitutes to tea in the market such as coffee, energy drinks and artificial tea. These substitutes can impact the demand negatively or positively depending on factors such as their quality and price.

\section{Determinants for supply}
The amount of products available in the market for sale at a specific price at a given point of time can be defined as the supply. Following is the list of determinants that play a major role in determining the supply of Talawakelle Tea Estates PLC.

\begin{itemize}
	\item {Technology}
	\item {The price of goods}
	\item {Cost of production}
	\item {Weather and other natural factors}
	\item {Taxes on products}
	\item {Subsidies}
\end{itemize}

\subsection{Technology}
Availability of modern technological solutions in the production and within the firm can greatly effect the supply because modern technology is capable of improving the overall efficiency and frequency of production.

\subsection{Cost of production}
The cost factor when producing goods can greatly affect the supply, because if the production costs are very high, the amount of goods that can be produced within a given time frame can be small compared to when production costs are low.

\subsection{Weather and other natural factors}
Changes in the climate and nature can greatly affect the capacity to create certain items. The reason for this is it can change the growth of tree plants and slow down the process of extracting leaves due to bad weather conditions. Therefore, this can affect the supply.

\subsection{Taxes on products}
Taxes on products, such as Value Added Tax (VAT), can have a direct effect on supply. The reason for this is when taxes are increased it can affect a producer’s decision to supply, and how much to supply.

\subsection{Subsidies}
Subsidies are funds given to a firm in order to increase the production and supply or reduce the price of their products. This can affect the production capability of a firm with respect to the supply.

\chapter{Government intervention}
Government policies regarding local markets and exports have a great impact on the operations of Talawakelle Tea Estates PLC. Tea industry has been one of the primary foreign income sources of Sri Lanka since the past and thus government has been continuously improving policies and procedures related to tea industry. The government have also been helping to develop community infrastructure through collaboration with governmental and non-governmental organizations.

The government have been helping with programs such as the "Immunisation Program" to conduct vaccinations for workers in tea estates covering illnesses such as BCG, Penta, Polio and MMR. These community healthcare programs have been contributing to the overall performance of workers and to improve the production of goods. As a result of these contributions and collaborations with the government there are fully equipped medical centres within estates along with ambulance services in key locations.

However, there is a risk of government policies including subsidies impacting supplier businesses and relationships with the company. Depending on the nature of the policy this can be either negative or positive. The recent Glyphosate and weedicide ban imposed by the government caused the production to slow down and the company had to reckon with high weeding cost and labour issues in up country regions.

\chapter{Resources}

\section{Fixed Costs}
Costs that do not vary with the volume of production are known as fixed costs. These costs stay the same regardless of goods or services are produced or not. As a result of this a company can not avoid fixed costs.

Following is a list of some fixed costs that can be seen in this company.

\begin{itemize}
	\item {Staff and executive salaries}
	\item {Insurance}
	\item {Deprecation of machines}
	\item {Equipment lease payment}
	\item {Utility payments}
\end{itemize}

\section{Variable Costs}
Costs that are associated with the number of goods or services provided are known as variable costs. These costs vary based on the production value of the company. Thus, a company has more control over these type of costs.


Following is a list of some variable costs that can be seen in this company.

\begin{itemize}
	\item {Direct materials}
	\item {Direct labour}
	\item {Commissions}
	\item {Variable overhead to the product}
	\item {Utility payments}
\end{itemize}


\chapter{Product pricing}
The company takes into account all factors affecting both demand price and supply when deciding product prices. Since Talawakelle Tea Estates PLC does not rely on third parties for the most important material which are tea laves, the company have been able to maintain a significantly low commodity cost.

Following is a list of most important factors the company considers when pricing a product.

\begin{itemize}
	\item {Costs}
	\item {Customers}
	\item {Positioning}
	\item {Competitors}
	\item {Profit}
\end{itemize}


\chapter{Market structure}
Talawakelle Tea Estates PLC has an oligopoly market structure due to the fact that there are only a handful of tea manufacturers in Sri Lanka.

Following is a list of tea manufactures in Sri Lanka.
\begin{itemize}
	\item {Venture Tea (Pvt) Ltd}
	\item {Ahinsa Tea}
	\item {Dilmah Tea}
	\item {Richy Group}
	\item {Halpé Tea}
\end{itemize}

Even in this competitive market, the company have been constantly gaining market share and customers over the past few years. The following charts indicate the growth of the company with respect to various factors in the market (Source: Annual report 2018/2019).

\begin{figure}[H]
	\centering
	\includegraphics[width=1\textwidth]{operations_review.png}
	\caption{Operations Review of Talawakelle Tea Estates PLC}
\end{figure}

\chapter{Import and export}
The imports of the company have been significantly low except for mechanical components and machinery and none of the imports are directly materials needed for the production.

Overall exports of products manufactured by the company posted a marginal decline due to negative growth happening in some Middle Eastern countries and Russia as a result of political uncertainties.

The following table shows tea export quantities from 2016 to 2018 and top ten countries.

\begin{figure}[H]
	\centering
	\includegraphics[width=1\textwidth]{tea_exports.png}
	\caption{Tea exports - Top Ten Countries 2016/2017/2018}
\end{figure}

As a result of Japaneese market uncertainty over Ceylon tea, mid-year unrewarding weather conditions and US sanctions over Iran adversely on tea exports, earnings in 2018 moderated after recoding the highest ever revenue. Even though there were problems, South East Asia and some major Middle Eastern countries have shown a positive interest, and it resulted in a better stability with good earnings.

The following chart shows tea export earnings from 2014 to 2018.

\begin{figure}[H]
	\centering
	\includegraphics[width=0.5\textwidth]{tea_export_earnings.png}
	\caption{Tea Export Earnings}
\end{figure}

When overall product performance is considered, bulk tea exports have a marginal decrease compared to previous years. The following table shows the tea product performance from 2016 to 2018.

\begin{figure}[H]
	\centering
	\includegraphics[width=1\textwidth]{tea_product_performance.png}
	\caption{Tea Product Performance}
\end{figure}


\chapter{Impact of COVID-19}
The tea industry and supply chain of Sri Lanka consists of producers, distributors, importers, national and international logistics systems and exporters who cater to both domestic and non-domestic markets. In this spectrum Talawakelle Tea Estates PLC mainly act as an exporter and a producer.

This pandemic has caused disruptions in the elements of tea industry such as production and exports causing various shocks in the trade. The following is a list of shocks  that can be seen in the tea supply chain as a result of COVID-19.

\begin{itemize}
	\item {Demand shock}
	\item {Trade shock}
	\item {Supply shock}
\end{itemize}

There is a good chance of the supply of labour-intensive agricultural products in low-income countries to be badly affected as a result of COVID-19. As a result of this, Sri Lankan tea industry have shown a significant decrease in production during early 2020 compared to past years. Even though there are other factors such as extreme weather conditions and low fertilizer applications, that contributed to this decline, the COVID-19 has a huge impact on the sudden decline. As a result of Sri Lanka primarily exporting black tea, there is a high chance for this decline to drop further.

The following chart showings the decline of quarterly tea production of Sri Lanka.
\begin{figure}[H]
	\centering
	\includegraphics[width=0.9\textwidth]{tea_production_2020.png}
	\caption{Quarterly Tea Production in Sri Lanka}
\end{figure}

When demand for the tea is considered, international price seem to fluctuate a lot due to changes in consumer preferences and purchasing power. Also, importing volumes of major buyers have shown a decrease. There has been a decline in the volume of shipping due to shrinking global trade as a result of drop in demand caused by COVID-19.

The following table from Markets Reports from the Tea Exporters Association shows the drop of importing volumes of major Ceylon Tea buyers.

\begin{table}[H]
	\centering
	\begin{tabular}{ |p{4cm}|p{6cm}| }
		\hline
		\bf{Country} & \bf{Change in importing volume} \\
		\hline
		Iraq         & -18\%                           \\
		Turkey       & -14\%                           \\
		Russia       & -6\%                            \\
		Syria        & -24\%                           \\
		\hline
	\end{tabular}
	\caption{Drop of importing volumes due to COVID-19}
\end{table}

However, a slight growth can be seen in the imports of major buyers such as Russia and Turkey in late 2020.

The demand seems to be shifting towards more herbal and health-based products due to the impact of COVID-19 and as a result of this the tea industry of Sri Lanka have been starting introduce new products such as different green tea variants.

\end{document}